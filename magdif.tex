%% LyX 2.2.2 created this file.  For more info, see http://www.lyx.org/.
%% Do not edit unless you really know what you are doing.
\documentclass[12pt,british,english]{article}
\usepackage[T1]{fontenc}
\usepackage[latin9]{inputenc}
\usepackage{geometry}
\geometry{verbose,tmargin=2cm,bmargin=2cm,lmargin=2cm,rmargin=2cm}
\setlength{\parskip}{\smallskipamount}
\setlength{\parindent}{0pt}
\usepackage{amsmath}
\usepackage{esint}
\usepackage{babel}
\begin{document}
\selectlanguage{british}%
\global\long\def\tht{\vartheta}
\global\long\def\ph{\varphi}
\global\long\def\balpha{\boldsymbol{\alpha}}
\global\long\def\btheta{\boldsymbol{\theta}}
\global\long\def\bJ{\boldsymbol{J}}
\global\long\def\bGamma{\boldsymbol{\Gamma}}
\global\long\def\bOmega{\boldsymbol{\Omega}}
\global\long\def\d{\text{d}}
\global\long\def\t#1{\text{#1}}
\global\long\def\m{\text{m}}
\global\long\def\bm{\text{\textbf{m}}}
\global\long\def\k{\text{k}}
\global\long\def\i{\text{i}}

\selectlanguage{english}%

\title{Numerical solution of magnetic differential equations}
\maketitle

\section*{Simple Example}

We would like to solve an equation of the type
\begin{align}
\dot{p}_{n}+inp_{n} & =q_{n}
\end{align}
for $p(s)$ with periodic boundary conditions at $s=0\dots2\pi$.
With the ansatz $p_{n}=\sum_{m}p_{mn}e^{ims}$ and the same for $q_{n}$
we obtain the analytical solution
\begin{align}
p_{mn} & =\frac{q_{mn}}{i(m+n)}.
\end{align}
A finite difference scheme yields
\begin{align}
\frac{p_{n}^{k+1}-p_{n}^{k}}{\Delta s}+\frac{1}{2}in(p_{n}^{k+1}+p_{n}^{k}) & =q_{n}^{k}.
\end{align}
A corresponding matrix with periodic boundary conditions is
\begin{align*}
\left(\begin{array}{cccc}
in/2-1/\Delta s & in/2+1/\Delta s\\
 & in/2-1/\Delta s & in/2+1/\Delta s\\
 &  & \dots\\
in/2+1/\Delta s &  &  & in/2-1/\Delta s
\end{array}\right)\boldsymbol{p} & =\boldsymbol{q}
\end{align*}
\begin{align*}
\left(\begin{array}{cccc}
in/2-1/\Delta s & in/2+1/\Delta s\\
in/2+1/\Delta s & in/2-1/\Delta s
\end{array}\right)\boldsymbol{p} & =\boldsymbol{q}
\end{align*}


\section*{Magnetic Differential Equation}

Magnetic differential equations arise from a number of problems in
plasma physics. We consider for example the magnetohydrodynamic equilibrium
\begin{align}
\nabla p & =\boldsymbol{J}\times\boldsymbol{B},
\end{align}
with pressure $p$, current $\boldsymbol{J}$ and magnetic field $\boldsymbol{B}$.
Scalar multiplication with $\boldsymbol{B}$ yields the homogenous
magnetic differential equation
\begin{align}
\boldsymbol{B}\cdot\nabla p & =0\,.
\end{align}
In the linear perturbation theory, a source term enters the right-hand
side with
\begin{align}
\boldsymbol{B}\cdot\nabla p & =q\,.
\end{align}
For an axisymmetric plasma in a tokamak, we reduce the dimensionality
by introducing cylindrical coordinates and an expansion in $\varphi$,
\begin{align}
p(R,Z,\ph) & =\sum_{n}p_{n}(R,Z)e^{in\ph}\,.
\end{align}
The remaining equation in the poloidal $RZ$ plane for each harmonic
are
\begin{align}
\boldsymbol{B}\cdot\nabla_{RZ}p_{n}+inB^{\varphi}p_{n} & =q_{n}\,.
\end{align}

or in components of coordinates $x^{k}$ with $k=1,2$ in the poloidal
plans,
\begin{align}
B^{k}\frac{\partial}{\partial x^{k}}p_{n}+inB^{\varphi}p_{n} & =q_{n}\,.
\end{align}

Generating $\boldsymbol{B}$ from the stream function $\psi=A_{\ph}$
we obtain
\begin{align}
B^{1} & =-\frac{1}{R\sqrt{g_{p}}}\frac{\partial\psi}{\partial x^{2}}\\
B^{2} & =\frac{1}{R\sqrt{g_{p}}}\frac{\partial\psi}{\partial x^{1}}
\end{align}
Here, $g_{p}$ is the metric determinant of the 2D metric tensor in
the poloidal plane. Using $\psi$ as one coordinate $x^{1}$, and
the distance $s$ in the poloidal direction with $ds=\sqrt{dR^{2}+dZ^{2}}$,
we have an orthogonal system with
\begin{align}
\hat{g}_{P} & =\left(\begin{array}{cc}
g_{\psi\psi}\\
 & 1
\end{array}\right)
\end{align}
This means that $\sqrt{g_{p}}=\sqrt{g_{\psi\psi}}=1/|\nabla\psi|$.
The transport law becomes
\begin{align}
\frac{1}{R\sqrt{g_{p}}}\frac{\partial p_{n}}{\partial s}+inB^{\varphi}p_{n} & =q_{n}\,.
\end{align}
This is a one-dimensional problem along the poloidal $\boldsymbol{B}$
direction. 

\section*{Finite Difference Method}

Multiplying by $-iR\sqrt{g_{p}}$ we obtain
\begin{align}
nR\sqrt{g_{p}}B^{\varphi}p_{n}-i\dot{p}_{n} & =-iR\sqrt{g_{p}}q_{n}\,.
\end{align}
Discretizing with a forward Euler method and evaluating averages at
the midpoints we obtain
\begin{align}
\frac{n}{2}\left(R^{k}\sqrt{g_{p}^{k}}B^{\varphi k}p_{n}^{k}+R^{k+1}\sqrt{g_{p}^{k+1}}B^{\varphi k+1}p_{n}^{k+1}\right) & -i\frac{p_{n}^{k+1}-p_{n}^{k}}{\Delta s^{k}}\\
 & =-\frac{1}{2}i\left(R^{k}\sqrt{g_{p}^{k}}q_{n}^{k}+R^{k+1}\sqrt{g_{p}^{k+1}}q_{n}^{k+1}\right)\,.
\end{align}
Coefficients should be filled into a sparse matrix and the discrete
equations solved e.g. by UMFPACK.

\section*{Analytical solution}

\begin{align*}
i(mB^{\vartheta}+nB^{\varphi})p_{mn} & =q_{mn}\\
p_{mn} & =\frac{q_{mn}}{i(mB^{\vartheta}+nB^{\varphi})}\,.
\end{align*}
We take circular flux surfaces in the large aspect ratio limit, such
that the scaling with $R$ vanishes and as coordinates minor $r$
and $\vartheta$.

In this case

\begin{align*}
B^{\tht} & =\frac{1}{R_{0}\sqrt{g_{P}}}\frac{\partial\psi}{\partial r}\,.
\end{align*}
We set $\psi=r^{2}/4$ so $\frac{\partial\psi}{\partial r}=|\nabla\psi|=r/2$.
Due to the circular flux surfaces, we have a orthogonal system and
$\sqrt{g_{P}}=r$, so $B^{\tht}=1/(2R_{0})$.

\section*{Perturbation in current density}

First variant: Use linear perturbation
\begin{align}
\boldsymbol{j}\times\boldsymbol{B} & \approx\boldsymbol{j}_{0}\times\boldsymbol{B}_{0}+\delta\boldsymbol{j}\times\boldsymbol{B}_{0}+\boldsymbol{j}_{0}\times\delta\boldsymbol{B}=c(\nabla p_{0}+\nabla\delta p),
\end{align}
resulting in
\begin{align}
\delta\boldsymbol{j}\times\boldsymbol{B}_{0} & =\delta\boldsymbol{j}_{\perp}\times\boldsymbol{B}_{0}=c\nabla\delta p-\boldsymbol{j}_{0}\times\delta\boldsymbol{B}.\label{eq:j times B0}
\end{align}
Second variant: Use derived expresion for $\delta\boldsymbol{j}_{\perp}$
with
\begin{align}
\delta\boldsymbol{j}_{\perp} & =j_{0\parallel}\frac{\delta\boldsymbol{B}_{\perp}}{B_{0}}-\frac{c\boldsymbol{h}_{0}\cdot\delta\boldsymbol{B}}{B_{0}^{2}}\boldsymbol{h}_{0}\times\nabla p_{0}+\frac{c}{B_{0}}\boldsymbol{h}_{0}\times\nabla\delta p.
\end{align}
Take cross product with $\boldsymbol{B}_{0}$. 
\begin{itemize}
\item First term
\begin{align}
j_{0\parallel}\frac{\delta\boldsymbol{B}_{\perp}}{B_{0}}\times\boldsymbol{B}_{0} & =\delta\boldsymbol{B}_{\perp}\times(j_{0\parallel}\boldsymbol{h}_{0})=\delta\boldsymbol{B}_{\perp}\times\boldsymbol{j}_{0\parallel}.
\end{align}
\item Second term
\begin{align}
-\frac{c\boldsymbol{h}_{0}\cdot\delta\boldsymbol{B}}{B_{0}^{2}}(\boldsymbol{h}_{0}\times\nabla p_{0})\times\boldsymbol{B}_{0} & =-\frac{\delta B_{\parallel}}{B_{0}}(\boldsymbol{h}_{0}\times(\boldsymbol{j}_{0}\times\boldsymbol{B}_{0}))\times\boldsymbol{h}_{0}\nonumber \\
 & =-\delta B_{\parallel}(\boldsymbol{j}_{0\perp}\times\boldsymbol{h}_{0})=\delta\boldsymbol{B}_{\parallel}\times\boldsymbol{j}_{0\perp}.
\end{align}
\item Third term
\begin{align*}
\frac{c}{B_{0}}(\boldsymbol{h}_{0}\times\nabla\delta p)\times\boldsymbol{B}_{0} & =c(\nabla\delta p-(\boldsymbol{h}_{0}\cdot\nabla\delta p)\boldsymbol{h}_{0})
\end{align*}
\end{itemize}
Summed up this yields
\begin{align}
\delta\boldsymbol{j}_{\perp}\times\boldsymbol{B}_{0} & =\delta\boldsymbol{B}\times\boldsymbol{j}_{0}-\delta\boldsymbol{B}_{\perp}\times\boldsymbol{j}_{0\perp}+c\nabla\delta p-c(\boldsymbol{h}_{0}\cdot\nabla\delta p)\boldsymbol{h}_{0}.\label{eq:jperp}
\end{align}
If Eq.~(\ref{eq:j times B0}) is fulfilled, the two extra terms must
cancel each other. Without the mentioned restriction
\begin{align}
(\delta\boldsymbol{B}\times\boldsymbol{j}_{0})_{\parallel} & =c\nabla_{\parallel}\delta p,
\end{align}
Eq.~(\ref{eq:jperp}) restricts only perpendicular components with
\begin{align}
\delta\boldsymbol{j}_{\perp}\times\boldsymbol{B}_{0} & =(\delta\boldsymbol{B}\times\boldsymbol{j}_{0})_{\perp}+c\nabla_{\perp}\delta p.
\end{align}


\section*{Finite Volume Methode}

Now we a similar problem using a FVM scheme. We write the conservative
form
\begin{align*}
\nabla\cdot(\boldsymbol{h}_{0}j_{\parallel n})+inh_{0}^{\varphi}j_{\parallel n} & =-\nabla\cdot\boldsymbol{j}_{\perp}^{\text{pol}}-inj_{\perp n}^{\varphi}\,.
\end{align*}

The divergence operator is defined via
\begin{align*}
\nabla\cdot\boldsymbol{u} & =\frac{1}{R\sqrt{g_{p}}}\frac{\partial}{\partial x^{k}}(R\sqrt{g_{p}}u^{k})\,.
\end{align*}
When working with $R,Z$ as coordinates in the poloidal plane, $\sqrt{g_{p}}=1$.
We multiply by $R$ to obtain
\begin{align*}
\frac{\partial}{\partial x^{k}}(Rh^{k}j_{\parallel n})+inRh^{\varphi}j_{\parallel n} & =-\frac{\partial}{\partial x^{k}}(Rj_{\perp n}^{k})-inRj_{\perp n}^{\varphi}\,.
\end{align*}
Integration over a triangle yields
\begin{align}
\oint\,Rj_{\parallel n}\boldsymbol{h}_{0}^{\text{pol}}\cdot\boldsymbol{n}d\Gamma+in\int Rh_{0}^{\varphi}j_{\parallel n}d\Omega & =-\oint\,R\boldsymbol{j}_{\perp n}^{\text{pol}}\cdot\boldsymbol{n}d\Gamma-in\int Rj_{\perp n}^{\ph}d\Omega\label{eq:fvmj-1}
\end{align}
where scalar products with $\boldsymbol{n}$ pointing towards the
outer normal vector of the edge are taken component-wise in $R$ and
$Z$. We assume a field-aligned mesh with $\boldsymbol{h}_{0}$ parallel
to edge no.~3. In- and outflux of the parallel current are only over
edges 1 and 2.

On edges 1 and 2 we sum up the flux from left- and right hand side
of Eq.~\ref{eq:fvmj-1} to use the flux of the total perturbed current
harmonic in the poloidal plane,
\begin{align*}
\boldsymbol{j}_{n}^{\text{pol}} & =j_{\parallel n}\boldsymbol{h}_{0}^{\text{pol}}+\boldsymbol{j}_{\perp n}^{\text{pol}},
\end{align*}
as an unknown:
\begin{align}
\int_{1,2}\,R\boldsymbol{j}_{n}^{\text{pol}}\cdot\boldsymbol{n}d\Gamma+in\int Rh_{0}^{\varphi}j_{\parallel n}d\Omega & =-\int_{3}\,R\boldsymbol{j}_{\perp n}^{\text{pol}}\cdot\boldsymbol{n}d\Gamma-in\int Rj_{\perp n}^{\ph}d\Omega.
\end{align}
In addition, we add terms with $j_{n}^{\ph}=h_{0}^{\varphi}j_{\parallel n}+j_{\perp n}^{\ph}$
together again to obtain
\begin{align}
\int_{1,2}\,R\boldsymbol{j}_{n}^{\text{pol}}\cdot\boldsymbol{n}d\Gamma+\int_{3}\,R\boldsymbol{j}_{\perp n}^{\text{pol}}\cdot\boldsymbol{n}d\Gamma+in\int Rj_{n}^{\varphi}d\Omega & =0.
\end{align}
To compute known quantities we use the linear equation of the perturbation
given by
\begin{align}
\boldsymbol{j}_{n}\times\boldsymbol{B}_{0} & =c(\nabla p_{n}+in\,p_{n}\nabla\varphi)-\boldsymbol{j}_{0}\times\boldsymbol{B}_{n}.
\end{align}
Again $\boldsymbol{B}_{n}$ is known from the last iteration of the
field solver, $p_{n}$ has been computed in the earlier step and $\boldsymbol{j}_{n}$
is unknown.

\subsection*{Cross-field term on edge 3}

For the term with the integral over edge 3 we use scalar multiplication
by \textbf{$\boldsymbol{e}_{\ph}=\frac{\partial\boldsymbol{R}}{\partial\ph}$
}that yields
\begin{align}
\boldsymbol{e}_{\ph}\cdot(\boldsymbol{j}_{n}\times\boldsymbol{B}_{0}) & =\boldsymbol{j}_{n}\cdot(\boldsymbol{B}_{0}^{\text{pol}}\times\boldsymbol{e}_{\ph}).
\end{align}

The poloidal part of the equilibrium field follows from the poloidal
flux (stream function) $\psi$ with
\begin{align}
\boldsymbol{B}_{0}^{\text{pol}} & =\nabla\psi\times\nabla\varphi.
\end{align}
We can use the double cross product
\begin{align}
(\nabla\psi\times\nabla\varphi)\times\boldsymbol{e}_{\ph} & =(\boldsymbol{e}_{\ph}\cdot\nabla\psi)\nabla\varphi-(\boldsymbol{e}_{\ph}\cdot\nabla\varphi)\nabla\psi=-\nabla\psi,
\end{align}
since the first term vanishes due to axisymmetry and the second term
yields unity for the inner product between basis vector and its reciprocal
vector. In addition we the fully poloidal vector potential $\boldsymbol{A}_{n}$
producing the non-axisymmetric harmonic $\boldsymbol{B}_{n}$ via
\begin{align}
B_{n}^{R}=\frac{in}{R}A_{nZ},\quad B_{n}^{Z} & =-\frac{in}{R}A_{nR},
\end{align}
and
\begin{align}
\boldsymbol{B}_{n}\times\boldsymbol{e}_{\ph} & =\frac{in}{R}(A_{nZ}\boldsymbol{e}_{R}\times\boldsymbol{e}_{\ph}-A_{nR}\boldsymbol{e}_{Z}\times\boldsymbol{e}_{\ph})\nonumber \\
 & =in(A_{nZ}\nabla Z+A_{nR}\nabla R)=in\boldsymbol{A}_{n}.
\end{align}

This results in 
\begin{align}
\boldsymbol{j}_{n}\cdot\nabla\psi & =-\left(in\,p_{n}-\boldsymbol{j}_{0}\cdot(\boldsymbol{B}_{n}\times\boldsymbol{e}_{\ph})\right)\nonumber \\
 & =-in\left(p_{n}-\boldsymbol{j}_{0}^{\text{pol}}\cdot\boldsymbol{A}_{n}\right).
\end{align}
Via normalisation via $|\nabla\psi|$, its orientation, and the edge
length/orientation, this term describing currents across flux surfaces
can be computed right away.

\subsection*{Volumetric source term}

For the computation of toroidal $j_{n}^{\varphi}$ in the element
volume we start again with
\begin{align}
\boldsymbol{j}_{n}\times\boldsymbol{B}_{0} & =c(\nabla p_{n}+in\,p_{n}\nabla\varphi)-\boldsymbol{j}_{0}\times\boldsymbol{B}_{n}.
\end{align}
with
\begin{align}
\boldsymbol{B}_{0} & =\nabla\psi\times\nabla\varphi+B_{0\ph}\nabla\ph.
\end{align}

By taking the inner product with $\boldsymbol{e}_{\psi}$ we obtain
\begin{align}
\boldsymbol{e}_{\psi}\cdot(\boldsymbol{j}_{n}\times\boldsymbol{B}_{0})=\boldsymbol{e}_{\psi}\cdot(\boldsymbol{j}_{n}\times(\nabla\psi\times\nabla\varphi)) & =\boldsymbol{j}_{n}\cdot((\nabla\psi\times\nabla\varphi)\times\boldsymbol{e}_{\psi})\nonumber \\
 & =\boldsymbol{j}_{n}\cdot\nabla\varphi=j_{n}^{\ph}
\end{align}
on the left-hand side. The right-hand side yields:
\begin{align}
\boldsymbol{e}_{\psi}\cdot(\nabla p_{n}+in\,p_{n}\nabla\varphi) & =\boldsymbol{e}_{\psi}\cdot\nabla p_{n}=\frac{\partial p_{n}}{\partial\psi}\\
\boldsymbol{e}_{\psi}\cdot(\boldsymbol{j}_{0}\times\boldsymbol{B}_{n}) & =\frac{\nabla\psi}{|\nabla\psi|^{2}}\cdot(B_{n\ph}\boldsymbol{j}_{0}^{\text{pol}}\times\nabla\ph-j_{0\ph}\boldsymbol{B}_{n}^{\text{pol}}\times\nabla\ph)\nonumber \\
 & =\frac{\boldsymbol{B}_{0}^{\text{pol}}}{|\nabla\psi|^{2}}\cdot(j_{0\ph}\boldsymbol{B}_{n}^{\text{pol}}-B_{n\ph}\boldsymbol{j}_{0}^{\text{pol}}).
\end{align}


\part*{Old}

Cross product with $\boldsymbol{e}_{\ph}$ yields
\begin{align}
B_{0\ph}\boldsymbol{e}_{\ph}\times(\boldsymbol{j}_{n}\times\nabla\ph) & =\boldsymbol{j}_{n}-j_{n\ph}\nabla\ph\\
\boldsymbol{e}_{\ph}\times(\boldsymbol{j}_{n}\times(\nabla\psi\times\nabla\varphi)) & =\boldsymbol{e}_{\ph}\times(j_{n}^{\ph}\nabla\psi-j_{n}^{\psi}\nabla\varphi)=j_{n}^{\ph}\boldsymbol{e}_{\ph}\times\nabla\psi\nonumber \\
 & =j_{n\ph}\nabla\ph\times\nabla\psi\nonumber 
\end{align}

\begin{itemize}
\item Scalar product with $\boldsymbol{B}_{0}^{\text{pol}}=\nabla\psi\times\nabla\ph$
yields
\begin{align}
\boldsymbol{B}_{0}^{\text{pol}}\cdot(\boldsymbol{j}_{n}\times\boldsymbol{B}_{0}^{\text{pol}}) & =0\\
B_{0\ph}(\nabla\psi\times\nabla\ph)\cdot(\boldsymbol{j}_{n}\times\nabla\ph) & =\frac{B_{0\ph}}{R^{2}}\boldsymbol{j}_{n}\cdot\nabla\psi
\end{align}
Right-hand side:
\begin{align*}
\boldsymbol{B}_{0}^{\text{pol}}\cdot(\nabla p_{n}+in\,p_{n}\nabla\varphi) & =\boldsymbol{B}_{0}^{\text{pol}}\cdot\nabla p_{n}\\
\boldsymbol{B}_{0}^{\text{pol}}\cdot(\boldsymbol{j}_{0}\times\boldsymbol{B}_{n}) & =(\nabla\psi\times\nabla\ph)\cdot(\boldsymbol{j}_{0}\times\boldsymbol{B}_{n})\\
 & =(\boldsymbol{j}_{0}\cdot\nabla\psi)(\boldsymbol{B}_{n}\cdot\nabla\ph)-(\boldsymbol{B}_{n}\cdot\psi)(\boldsymbol{j}_{0}\cdot\nabla\ph)
\end{align*}
\end{itemize}
We need to solve
\begin{align}
\nabla\cdot\delta\boldsymbol{j} & =0,\\
\nabla\delta p & =\frac{1}{c}\left(\delta\boldsymbol{j}\times\boldsymbol{B}_{0}+\boldsymbol{j}_{0}\times\delta\boldsymbol{B}\right)
\end{align}
for $\delta\boldsymbol{j}$. Splitting into toroidal and poloidal
parts for a single harmonic in $\varphi$ we obtain
\begin{align}
\nabla\cdot\boldsymbol{j}_{n}^{\text{pol}}+in\,j_{n}^{\ph} & =0.
\end{align}
The second equation reads
\begin{align}
\boldsymbol{j}_{n}\times\boldsymbol{B}_{0} & =c(\nabla p_{n}+in\,p_{n}\nabla\varphi)-\boldsymbol{j}_{0}\times\boldsymbol{B}_{n}.
\end{align}
The toroidal part of the cross product is
\begin{align}
(\boldsymbol{j}_{n}\times\boldsymbol{B}_{0})_{\ph} & =R(j_{n}^{Z}B_{0}^{R}-j_{n}^{R}B_{0}^{Z})=R\sqrt{g_{P}}j_{n}^{\psi}B_{0}^{\text{pol}}\\
 & =in\,p_{n}-(\boldsymbol{j}_{0}\times\boldsymbol{B}_{n})_{\varphi}\\
\Rightarrow j_{n}^{\psi} & \text{on flux surface edge}\nonumber 
\end{align}
On one triangle edge
\begin{align*}
(\boldsymbol{j}_{n}\times\boldsymbol{B}_{0})_{\parallel} & =R(j_{n\perp}B_{0}^{\ph}-j_{n}^{\ph}B_{0}^{\perp})\\
 & \approx c\frac{p_{2}-p_{1}}{l}-(\boldsymbol{j}_{0}\times\boldsymbol{B}_{n})_{\parallel}
\end{align*}


\section*{Finite Volume Method}

The divergence operator is defined via
\begin{align*}
\nabla\cdot\boldsymbol{u} & =\frac{1}{R\sqrt{g_{p}}}\frac{\partial}{\partial x^{k}}(R\sqrt{g_{p}}u^{k})\,.
\end{align*}
When working with $R,Z$ as coordinates in the poloidal plane, $\sqrt{g_{p}}=1$.
We multiply by $R$ to obtain
\begin{align*}
\frac{\partial}{\partial x^{k}}(Rh^{k}j_{\parallel n})+inRh^{\varphi}j_{\parallel n} & =-\frac{\partial}{\partial x^{k}}(Rj_{\perp n}^{k})-inRj_{\perp n}^{\varphi}\,.
\end{align*}
Integration over a triangle yields
\begin{align}
\oint\,Rj_{\parallel n}\boldsymbol{h}\cdot\boldsymbol{n}d\Gamma+in\int Rh^{\varphi}j_{\parallel n}d\Omega & =-\oint\,R\boldsymbol{j}_{\perp n}^{\text{pol}}\cdot\boldsymbol{n}d\Gamma-in\int Rj_{\perp n}^{\ph}d\Omega\label{eq:fvmj}
\end{align}
where scalar products with $\boldsymbol{n}$ pointing towards the
outer normal vector of the edge are taken component-wise in $R$ and
$Z$. We assume a field-aligned mesh with $\boldsymbol{h}$ parallel
to edge no.~3. The in- and outflux are over edges 1 and 2.

\section*{General Finite Volume Method}

Eq.~(\ref{eq:fvmj}) is of the form
\begin{align*}
\oint\,u\,\boldsymbol{h}\cdot\boldsymbol{n}d\Gamma+in\int u\,h^{\varphi}d\Omega & =-\oint\,\boldsymbol{v}\cdot\boldsymbol{n}d\Gamma-in\int w\,d\Omega
\end{align*}
with $u=Rj_{\parallel n}$, $\boldsymbol{v}=R\boldsymbol{j}_{\perp n}^{\text{pol}}$
and $w=Rj_{\perp n}^{\ph}$. We approximate the flux by a flux value
times edge length. Since the mesh is field-aligned, only two of the
three triangle edges play a role for fluxes and we can write
\begin{align}
\oint\,u\,\boldsymbol{h}\cdot\boldsymbol{n}d\Gamma & \approx U_{1}+U_{2}=u_{1}l_{1}+u_{2}l_{2},
\end{align}
where
\begin{align}
U_{1} & =\int_{1}u\boldsymbol{h}\cdot\boldsymbol{n}\d\Gamma_{1}.
\end{align}
and so on. In the second term we can use
\begin{align*}
\int u\,h^{\varphi}d\Omega & \approx\frac{u_{1}+u_{2}}{2}h^{\varphi}S
\end{align*}
where $S$ is the surface of the triangle.

For $\boldsymbol{v}$ the normal components to the edges (fluxes through
edges) are required via
\begin{align}
\oint\,\boldsymbol{v}\cdot\boldsymbol{n}d\Gamma & \approx V_{1}+V_{2}+V_{3}\nonumber \\
 & =\boldsymbol{v}_{1}\cdot\boldsymbol{n}_{1}l_{1}+\boldsymbol{v}_{2}\cdot\boldsymbol{n}_{2}l_{2}+\boldsymbol{v}_{3}\cdot\boldsymbol{n}_{3}l_{3}
\end{align}

Central difference scheme:
\begin{align*}
\dot{p}_{H} & =\frac{\Delta s^{k-1}}{\Delta s^{k}(\Delta s^{k}+\Delta s^{k-1})}p^{k+1}+\frac{\Delta s^{k}-\Delta s^{k-1}}{\Delta s^{k}\Delta s^{k-1}}p^{k}-\frac{\Delta s^{k}}{\Delta s^{k-1}(\Delta s^{k}+\Delta s^{k-1})}p^{k-1}
\end{align*}

In harmonics in $\varphi$ this becomes
\begin{align}
\boldsymbol{B}\cdot\nabla_{RZ}p_{n}+inB^{\varphi}p_{n} & =s_{n}\,.
\end{align}
If we take no toroidicity and harmonic RHS term with poloidal harmonic
$m$ we obtain
\begin{align*}
i(mB^{\vartheta}+nB^{\varphi})p_{mn} & =s_{mn}\\
p_{mn} & =\frac{s_{mn}}{i(mB^{\vartheta}+nB^{\varphi})}\,.
\end{align*}
Without toroidicity:
\begin{align*}
\sqrt{g} & =r\\
B^{\vartheta} & =\frac{1}{r}\partial_{r}A_{\varphi}
\end{align*}
So for $A_{\varphi}=r^{2}/4$ we get $B^{\vartheta}=1/2$. Furthermore,
we choose $B^{\varphi}=1$. We have
\begin{align*}
R & =R_{0}+r\cos\vartheta\\
Z & =r\sin\vartheta\\
\\
\frac{\partial R}{\partial r} & =(R-R_{0})/r\\
\frac{\partial Z}{\partial r} & =Z/r\\
\\
\frac{\partial R}{\partial\vartheta} & =-z\\
\frac{\partial Z}{\partial\vartheta} & =R-R_{0}\\
\\
B^{R} & =\frac{\partial R}{\partial\vartheta}B^{\vartheta}=-z/2\\
B^{Z} & =\frac{\partial Z}{\partial\vartheta}B^{\vartheta}=(R-R_{0})/2
\end{align*}
In real and imaginary parts this is
\begin{align}
\boldsymbol{B}\cdot\nabla_{RZ}(\Re p_{n}+i\Im p_{n})+inB^{\varphi}(\Re p_{n}+i\Im p_{n}) & =(\Re s_{n}+i\Im s_{n})\,\\
\boldsymbol{B}\cdot\nabla_{RZ}\Re p_{n}-nB^{\varphi}\Im p_{n} & =\Re s_{n}\,\\
\boldsymbol{B}\cdot\nabla_{RZ}\Im p_{n}+nB^{\varphi}\Re p_{n} & =\Im s_{n}\,
\end{align}
Combining
\begin{align*}
\boldsymbol{B}\cdot\nabla_{RZ}(\boldsymbol{B}\cdot\nabla_{RZ}\Re p_{n})-nB^{\varphi}(\Im s_{n}-nB^{\varphi}\Re p_{n}) & =\boldsymbol{B}\cdot\nabla_{RZ}\cdot\Re s_{n}\,\\
\boldsymbol{B}\cdot\nabla_{RZ}(\boldsymbol{B}\cdot\nabla_{RZ}\Im p_{n})+nB^{\varphi}(\Re s_{n}+nB^{\varphi}\Im p_{n}) & =\boldsymbol{B}\cdot\nabla_{RZ}\cdot\Im s_{n}\,
\end{align*}
In Flat space:
\begin{align*}
B^{R}\partial_{R}(\boldsymbol{B}\cdot\nabla_{RZ}\Re p_{n})+B^{Z}\partial_{Z}(\boldsymbol{B}\cdot\nabla_{RZ}\Re p_{n}) & \,\\
=(B^{R}\partial_{R}+B^{Z}\partial_{Z})(B^{R}\partial_{R}+B^{Z}\partial_{Z})\Re p_{n}\\
=\left((B^{R})^{2}\partial_{R}^{2}+2B^{R}B^{Z}\partial_{R}\partial_{Z}+\left(B^{Z}\right)^{2}\partial_{Z}^{2}\right)\Re p_{n}
\end{align*}
This equation is parabolic and not, as such, suited for FEM.

New:
\begin{align*}
\boldsymbol{B}\cdot\nabla_{RZ}(\boldsymbol{B}\cdot\nabla_{RZ}\Re p_{n}) & =\nabla_{RZ}\cdot(\boldsymbol{B}(\boldsymbol{B}\cdot\nabla_{RZ}\Re p_{n}))\\
 & =\nabla_{RZ}\cdot(\boldsymbol{B}\nabla_{RZ}\cdot(\boldsymbol{B}\Re p_{n}))
\end{align*}
In real and imaginary parts this is
\begin{align*}
i(mB^{\vartheta}+nB^{\varphi})(\Re p_{mn}+i\Im p_{mn}) & =(\Re s_{mn}+i\Im s_{mn})\\
\Im p_{mn} & =-\Re s_{mn}/(mB^{\vartheta}+nB^{\varphi})\\
\Re p_{mn} & =\Im s_{mn}/(mB^{\vartheta}+nB^{\varphi})
\end{align*}
We have
\begin{align*}
s & =\sum_{n}s_{n}(\vartheta)e^{in\varphi}=\sum_{mn}s_{mn}e^{i(m\vartheta+n\varphi)}\\
\\
 & =\sum_{n}(\Re s_{n}+i\Im s_{n})(\cos n\varphi+i\sin n\varphi)\\
 & =\sum_{n}(\Re s_{n}\cos n\varphi-\Im s_{n}\sin n\varphi)+i(\Re s_{n}\sin n\varphi+\Im s_{n}\cos n\varphi)\\
\\
 & =\sum_{mn}(\Re s_{mn}+i\Im s_{mn})(\cos(m\vartheta+n\varphi)+i\sin(m\vartheta+n\varphi))\\
 & =\sum_{mn}(\Re s_{mn}\cos(m\vartheta+n\varphi)-\Im s_{mn}\sin(m\vartheta+n\varphi))\\
 & +i(\Re s_{mn}\sin(m\vartheta+n\varphi)+\Im s_{mn}\cos(m\vartheta+n\varphi))\\
\\
s_{n} & =s_{mn}e^{im\vartheta}=(\Re s_{mn}+i\Im s_{mn})(\cos m\vartheta+i\sin m\vartheta)\\
 & =\Re s_{mn}\cos m\vartheta-\Im s_{mn}\sin m\vartheta+i(\Re s_{mn}\sin m\vartheta+\Im s_{mn}\cos m\vartheta)
\end{align*}
Test:
\begin{align*}
s & =\Im s_{mn}(\cos(m\vartheta+n\varphi)-i\sin(m\vartheta+n\varphi))\\
s_{n} & =s_{mn}e^{im\vartheta}=\Im s_{mn}(-\sin m\vartheta+i\cos m\vartheta)\\
 & =\\
\\
\Re p_{mn} & =\Im s_{mn}/(mB^{\vartheta}+nB^{\varphi})\\
\\
p_{n} & =\Re p_{mn}(\cos m\vartheta+i\sin m\vartheta)
\end{align*}


\section*{Pseudotoroidal coordinates}

\begin{align*}
R & =R_{0}+r\cos\vartheta\\
Z & =r\sin\tht\\
\\
\boldsymbol{e}_{r} & =\frac{\partial R}{\partial r}\boldsymbol{e}_{R}+\frac{\partial Z}{\partial r}\boldsymbol{e}_{Z}\\
 & =\boldsymbol{e}_{R}\cos\vartheta+\boldsymbol{e}_{Z}\sin\tht\\
\\
\boldsymbol{e}_{\tht} & =\frac{\partial R}{\partial\tht}\boldsymbol{e}_{R}+\frac{\partial Z}{\partial\tht}\boldsymbol{e}_{Z}\\
 & =-\boldsymbol{e}_{R}r\sin\vartheta+\boldsymbol{e}_{Z}r\cos\tht
\end{align*}

\end{document}
